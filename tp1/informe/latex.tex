\documentclass[a4paper,10pt]{article}

\usepackage[ansinew]{inputenc}
\usepackage{listings}

\lstset{ %
language=C,                			% choose the language of the code
basicstyle=\footnotesize,       		% the size of the fonts that are used for the code
numbers=left,                   			% where to put the line-numbers
numberstyle=\footnotesize,     	% the size of the fonts that are used for the line-numbers
stepnumber=1,                  		% the step between two line-numbers. If it is 1 each line will be numbered
numbersep=5pt,                  		% how far the line-numbers are from the code
showspaces=false,               		% show spaces adding particular underscores
showstringspaces=false,         	% underline spaces within strings
showtabs=false,                		% show tabs within strings adding particular underscores
frame=single,           			% adds a frame around the code
tabsize=2,          				% sets default tabsize to 2 spaces
captionpos=b,           			% sets the caption-position to bottom
breaklines=true,        			% sets automatic line breaking
breakatwhitespace=false,   		% sets if automatic breaks should only happen at whitespace
escapeinside={\%*}{*)}          		% if you want to add a comment within your code
}

\begin{document}

\title{Trabajo Pr\'actico nro.~1: assembly MIPS} 

\author{	
Luis Arancibia, \textit{Padr\'on XXXXX} 	\\	\texttt{$aran.com.ar@gmail.com$}	\\[2.5ex] 	
Matias Cerrotta, \textit{Padr\'on 89992} 	\\	\texttt{$matias.cerrotta@gmail.com$} 		\\[2.5ex]
Gabriel La Torre, \textit{Padr\'on XXXXX}	\\	\texttt{$latorregab@gmail.com$}			\\[2.5ex]
\normalsize{2do.~Cuatrimestre de 2015}		\\
\normalsize{66.20 Organizaci\'on de Computadoras  $-$ Pr\'actica Martes}	\\
\normalsize{Facultad de Ingenier\'ia, Universidad de Buenos Aires.}			\\
}

\date{}
\maketitle
\thispagestyle{empty}   % quita el nro en la primer pagina
\pagebreak

\begin{abstract}
Este trabajo pr\'actico nro.~1 busca familiarizar a los alumnos con el conjunto de instrucciones MIPS y el concepto de ABI, extendiendo un programa que resuelva el problema piloto que se presentar\'a m\'as abajo. 

Se mostrar\'a el c\'odigo C y el c\'odigo Assembly generado para la correcta resoluci\'on.

El programa correr\'a tanto en NetBSD/pmax, usando el simulador GXemul provisto por la c\'atedra, como en la versi\'on de Linux (Knoppix, RedHat, Debian, Ubuntu) usada para correr el simulador, Linux/i386.

\end{abstract} 
\pagebreak


\section{Introducci\'on}

El programa, a escribir en lenguaje C, deber\'a multiplicar matrices de n\'umeros reales, representados en punto flotante de doble precisi\'on. Las matrices a multiplicar ingresar\'an por entrada est\'andar (\textit{stdin}), donde cada l\'inea describe una matriz completa en formato de texto.
\newline \newline
NxM a1,1 a1,2 ... a1,M a2,1 a2,2 ... a2,M ... aN,1 aN,2 ... aN,M
\newline \newline
La l\'inea anterior representa a la matriz A de N filas y M columnas. Los elementos de la matriz A son los a x,y , siendo x e y los indices de fila y columna respectivamente 1 . El fin de l\'inea es el caracter newline. Los componentes de la l\'nea est\'an separados entre s\'i por uno o m\'as espacios. El formato de los n\'umeros en punto flotante son los que corresponden al especificador de conversi\'on g de printf.

Por ejemplo, dada la siguiente matriz:
\newline
1 2 3
\newline
4 5 6
\newline \newline
Su representaci\'on ser\'ia:
\newline
2x3 1 2 3 4 5 6
\newline \newline
Por cada par de matrices que se presenten en su entrada, el programa deber\'a multiplicarlas y presentar el resultado por su salida es\'andar (stdout) en el mismo formato presentado anteriormente, hasta que llegue al final del archivo de entrada (EOF). Ante un error, el progama
deber\'a informar la situaci\'on inmediatamente (por stderr) y detener su ejecuci\'on. Tener en cuenta que tambi\'en se condidera un error que a la entrada se presenten matrices de dimensiones incompatibles entre s\'i para su multiplicaci\'on.



\pagebreak


\section{Documentaci\'on}

\subsection{Compilaci\'on}
El programa se compilar\'a con la siguiente instrucci\'on para utilizar la implementaci\'on en C:
\begin{center} 
\fbox{gcc -Wall -O0 -o tp1 tp1.c multiplicar.c}
\end{center}

Y con la siguiente instrucci\'on para la implementaci\'on en Assembly:
\begin{center} 
\fbox{gcc -Wall -O0 -o tp1 tp1.c multiplicar.S}
\end{center}

Los tests se ejecutar\'an con el siguiente script:
\begin{center} 
\fbox{./tests.sh}
\end{center}

Ejemplo de la salida de ejecuci\'on:
\begin{verbatim}
$ ./tests.sh
	Tests #0 success_normal: 						OK
	Tests #1 success_espacios: 						OK
	Tests #2 error_dimension: 						OK
	Tests #3 error_dimension_caracter_invalido: 	OK
	Tests #4 error_dimension_cero: 					OK
	Tests #5 error_matriz1: 						OK
	Tests #6 error_matriz2: 						OK
\end{verbatim}

\subsection{Utilizaci\'on}
Opciones de ejecuci\'on:
\begin{verbatim}
	-h, --help		Print this information.
	-V, --version	Print version and quit.
\end{verbatim}

Ejemplos de ejecuci\'on:
\begin{verbatim}
Examples:
	./tp1 -h
	./tp1 -V
	./tp1 < in_file > out_file
	./tp1 < in.txt > out.txt
	cat in.txt | tp0 > out.txt

\end{verbatim}

A continuaci\'on se presenta un ejemplo de prueba:
\begin{verbatim}
	$ cat in.txt
	2x3 1 2 3 4 5 6.1
	3x2 1 0 0 0 0 1
	3x3 1 2 3 4 5 6.1 3 2 1
	3x1 1 1 0

	$ cat in.txt | ./tp0
	2x2 1 3 4 6.1
	3x1 3 9 5
\end{verbatim}
\pagebreak


\section{Casos de pruebas}
Se crearon los siguientes casos de pruebas:
\begin{enumerate}
\item Caso normal.	
\item Utilizando espacios entre elementos de matriz.
\item Dimensiones de matrices incompatibles.
\item Dimensiones de matrices con valores no alfanum\'ericos.
\item Dimensiones de matrices con valores inv\'alidos.
-\item Elementos de m\'as con respecto a la dimensi\'on.
\item Elementos de menos con respecto a la dimensi\'on.
\end{enumerate}				
\pagebreak


\section{C\'odigo fuente}

\begin{lstlisting}
#include <stdio.h>
#include <stdlib.h>
#include <getopt.h>
#include <string.h>
#include "multiplicar.h"

#define MAX_LINE_LENGTH 512
#define MAX_DIMENSION_LENGTH 4
#define EXIT_OK 0
#define EXIT_ERROR 1

enum ACCION {
	EMPTY,
	HELP,
	VERSION,
	MULTIPLICAR,
	ERROR
};

void multiplicarMatriz();
enum ACCION procesarArgumentos(int argc, char** argv);

int main(int argc, char **argv) {
	enum ACCION comando = procesarArgumentos(argc, argv);
	switch (comando) {
		case HELP:
			printf("Usage:\n"
			"\ttp0 -h\n"
			"\ttp0 -V\n"
			"\ttp0 < in_file > out_file\n"
			"Options:\n"
			"\t-V, --version\tPrint version and quit.\n"
			"\t-h, --help\tPrint this information.\n"
			"Examples:\n"
			"\ttp0 < in.txt > out.txt\n"
			"\tcat in.txt | tp0 > out.txt\n");
			break;
		case VERSION:
			printf("tp0 v0.1\n");
			break;
		case MULTIPLICAR:
			multiplicarMatriz();
			break;
		case ERROR:
		default:
			return EXIT_ERROR;
	}
	return EXIT_OK;
}


void imprimirMatriz(double* matriz, int filas, int columnas) {
	printf("%dx%d ", filas, columnas);
	int i;
	for (i = 0; i < filas * columnas; i++) {
			printf("%g ", matriz[i]);
	}
	printf("\n");
}

// Devuelve null sino hay memoria
char* append(char* original, int originalSize, char* toAppend) {
	if (original == NULL) {
		original = malloc(sizeof(toAppend));
	}	
	if (original == NULL) {
		//printf("out of memory\n");
		return NULL;
	}
	original = (char *) realloc(original, (originalSize + 1) * sizeof(char));
	if (original == NULL) {
		fprintf(stderr, " No hay más memoria.\n");
		return NULL;
	}
	original[originalSize] = *toAppend;
	return original;
}


void leerDimension(int* filas, int* columnas) {
	int newChar;
	char c;
	char* buffer = (char *) malloc(sizeof(char) *  MAX_DIMENSION_LENGTH);
	int i = 0, total = 0;
	while((newChar = getchar()) != EOF) {
		c = (char) newChar;

		if (c == 'x') {
			*filas = atoi(buffer);
			memset(buffer,0,strlen(buffer));
			i = 0;
			continue;
		}

		if(c == ' ') {
			*columnas = atoi(buffer);
			break;
		}

		if (c != '\n' && (c < '0' || c > '9')) {
			free(buffer);
			fprintf(stderr, "Dimension incorrecta.\n");
			exit(EXIT_ERROR);
		}

		if(total >= MAX_DIMENSION_LENGTH) {
			char* oldBuffer = buffer;
			buffer = append(buffer, total,&c);
			if (buffer == NULL) {
				free(oldBuffer);
				fprintf(stderr, " No hay más memoria.\n");
				exit(EXIT_ERROR);
			}
		} else {			
			buffer[i] = c;
		}
		i++;
		total++;
	}
	free(buffer);

	if (*filas == 0 || *columnas == 0) {
		fprintf(stderr, "Dimension incorrecta.\n");
		exit(EXIT_ERROR);
	}

	if (newChar == EOF)
		exit(EXIT_OK);
}


void leerMatriz(double* matriz, int filas, int columnas) {
	int newChar;
	char c;
	char* buffer = (char *) malloc(sizeof(char) * MAX_LINE_LENGTH);
	int i = 0, total = 0, elementos = 0, fila = 0, columna = 0;
	while((newChar = fgetc(stdin)) != EOF) {
		if (newChar == EOF) {
			break;
		}
		c = (char) newChar;
		total++;
		
		if (elementos >= filas * columnas) {
			fprintf(stderr, "%s\n", "Dimension no compatible con datos de matriz.");
			free(buffer);
			exit(EXIT_ERROR);
		}

		if (c == ' ' || c == '\n') {
			// elimino espacios consecutivos
			if (strlen(buffer) == 0)
				continue;

			elementos++;
			matriz[(fila*columnas)+columna] = atof(buffer);
			memset(buffer,0,strlen(buffer));
			i = 0;
			if (columna >= columnas - 1) {
				fila++;
				columna = 0;
			} else {
				columna++;
			}

			if(c == '\n')
				break;

			continue;
		}

		if(total >= MAX_LINE_LENGTH) {
			char* oldBuffer = buffer;
			buffer = append(buffer, total, &c);
			if (buffer == NULL) {
				free(oldBuffer);
				fprintf(stderr, "%s\n", "No hay más memoria para guardar la matriz.");
			}
			exit(EXIT_ERROR);
		}

		buffer[i] = c;
		i++;
	}
	free(buffer);
	
	if (elementos < filas * columnas) {
		fprintf(stderr, "%s\n", "Dimension no compatible con datos de matriz.");
		exit(EXIT_ERROR);
	}
	
	if (newChar == EOF)
		exit(EXIT_OK);
}


double* crearMatriz(int filas, int columnas) {
	double* matriz = (double*) malloc(sizeof(double) * filas * columnas);
	return matriz;
}


void multiplicarMatriz() {
	
	while (1) {
		int filasA, columnasA;
		leerDimension(&filasA, &columnasA);
		double* matrizA = crearMatriz(filasA, columnasA);
		leerMatriz(matrizA, filasA, columnasA);

		int filasB, columnasB;
		leerDimension(&filasB, &columnasB);
		double* matrizB = crearMatriz(filasB, columnasB);
		leerMatriz(matrizB, filasB, columnasB);

		if (columnasA != filasB) {
			fprintf(stderr, "%s\n", "Dimensiones incorrectas de matrices.");
			free(matrizA);
			free(matrizB);
			exit(EXIT_ERROR);
		}

		double* matrizC = crearMatriz(filasA, columnasB);
		
		multiplicarMatrices(filasA, matrizB, matrizC, matrizA, 
		 columnasB, columnasA);
		
		imprimirMatriz(matrizC, filasA, columnasB);

		free(matrizA);
		free(matrizB);
		free(matrizC);
	}
}


enum ACCION procesarArgumentos(int argc, char** argv) {
	// valores por defecto
	enum ACCION comando = EMPTY;
	
	 /* La funcion getopt obtiene el siguiente argumento especificado por argc y argv
	 * mas info: http://www.gnu.org/software/libc/manual/html_node/Using-Getopt.html#Using-Getopt
	 * La cadena "hVbri:o:" indica que h, V no tienen argumentos.
	 */
	int c;
	while ((c = getopt(argc, argv, "hV")) != -1) {
		switch (c) {
			case 'h':
				comando = HELP;
				break;
			case 'V':
				comando = VERSION;
				break;
			default:
				comando = ERROR;
				break;
		}
	}

	if (comando == EMPTY)
		comando = MULTIPLICAR;

	return comando;
}
\end{lstlisting}
\pagebreak
\begin{lstlisting}
void multiplicarMatrices(int filasA, double* matrizB, double* resultado, double* matrizA, int columnasB, int columnasA) {
	int i, j, k;
	for (i = 0; i < filasA; ++i) {
	  	for (j = 0 ; j < columnasB ; ++j) {
	  		resultado[i*columnasB+j]=0;
	      	for ( k = 0; k < columnasA; ++k) {
	      		resultado[i*columnasB+j] = (resultado[i*columnasB+j] + (matrizA[i*columnasA+k] * matrizB[k*columnasB+j]));
	    	}
		}
	}
}
\end{lstlisting}
\pagebreak


\section{C\'odigo $MIPS^{TM}$}
\begin{lstlisting}
#include <mips/regdef.h>


                # a0 cant filas A
                # a1 puntero a matriz B
                # a2 puntero a matriz C
                # a3 puntero a matriz A

                .text
                .globl  multiplicarMatrices
                .ent multiplicarMatrices

multiplicarMatrices:
        move t0, a0 # lw t0, cantf_c_i
        lw t1, 16(sp)# lw t1, cantc_c_j
        lw t2, 20(sp) # lw t2, cantc_a_k



        #no hago una estiqueta para i porque nunca volveremos a hacer i =0 para
        #Las filas
        li s0, 0 # i = 0; initialize 1st for loop
loop1:
        # Al contrario para las columnas de B se hace una corrida por cada fila
        #Varias veces a este punto
        li s1, 0 # j = 0; restart 2nd for loop
loop2:
        # Por cada valor de C deberemos iterar sobre todos los valores de K
        #
        li s2, 0 # k = 0; restart 3rd for loop

#Operaciones con C
        mul t3, s0, t1  # en t3 la cantidad de columnas de c
        addu t3, t3, s1 # en t3 sumo el valor de j-> la posicion en el array
        sll t3, t3, 3 #  el offset en bytes en el array
        addu t3, a2, t3 # cargo en t3 la direccion de C{i}{j}
        sw zero, 0(t3) # inicializo C{i}{j} con cero
	l.d $f4, 0(t3)
        #add t4, zero, zero # inicializo en t4 en cero que será el valor de C{i}
		

#Operaciones con A
loop3:
        mul t7, s0, t2  # en t7 la cantidad de columnas de A "K"
        addu t7, t7, s2 #  ahora le sumamos K y obtenemos la posicion en el arra
        sll t7, t7, 3 #  multiplcamos por 4 y obtenemos el offset de A[i][k]
        addu t7, a3, t7 # obtenemos la direccion de A[i][k]
        l.d $f8, 0(t7) # guardamos en t8 el valor de A[i][k]
        
# Ahora con B

        mul t5, s2, t1 # t5 = k * 4 (size of row of b)
        addu t5, t5, s1 # t5 = k * size(row) + j
        sll t5, t5, 3 # t5 = byte offset off [k][j]
        addu t5, a1, t5 # t5 = byte address of b[k][j]
        l.d $f6, 0(t5)



mul.d $f6, $f8, $f6 # t6 = a[i][k] * b[k][j]
add.d $f4, $f4, $f6 # t4 = c[i][j] + a[i][k] * b[k][j]


s.d $f4, 0(t3) # c[i][j] = t4  guardo el valor calculado en c[i][j]

addiu s2, s2, 1 # k = k + 1
bne s2, t2, loop3 #if (k != 4) go to loop3

addiu s1, s1, 1 # j = j + 1
bne s1, t1, loop2 # if (j != 4) go to loop2

addiu s0, s0, 1 # i = i + 1
bne s0, t0, loop1 # if (i != 32) go to loop1

jalr ra

.end multiplicarMatrices
\end{lstlisting}
\pagebreak

\end{document}
