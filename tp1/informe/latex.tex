\documentclass[a4paper,10pt]{article}

\usepackage[ansinew]{inputenc}
\usepackage{listings}

\lstset{ %
language=C,                			% choose the language of the code
basicstyle=\footnotesize,       		% the size of the fonts that are used for the code
numbers=left,                   			% where to put the line-numbers
numberstyle=\footnotesize,     	% the size of the fonts that are used for the line-numbers
stepnumber=1,                  		% the step between two line-numbers. If it is 1 each line will be numbered
numbersep=5pt,                  		% how far the line-numbers are from the code
showspaces=false,               		% show spaces adding particular underscores
showstringspaces=false,         	% underline spaces within strings
showtabs=false,                		% show tabs within strings adding particular underscores
frame=single,           			% adds a frame around the code
tabsize=2,          				% sets default tabsize to 2 spaces
captionpos=b,           			% sets the caption-position to bottom
breaklines=true,        			% sets automatic line breaking
breakatwhitespace=false,   		% sets if automatic breaks should only happen at whitespace
escapeinside={\%*}{*)}          		% if you want to add a comment within your code
}

\begin{document}

\title{Trabajo Pr\'actico nro.~0: Infraestructura b\'asica} 

\author{	Matias Cerrotta, \textit{Padr\'on 89992} 		\\	\texttt{$matias.cerrotta@gmail.com$}	\\[2.5ex] 	
		\normalsize{2do.~Cuatrimestre de 2015}		\\
	        	\normalsize{66.20 Organizaci\'on de Computadoras  $-$ Pr\'actica Martes}	\\
          	\normalsize{Facultad de Ingenier\'ia, Universidad de Buenos Aires.}			\\
	}

\date{}
\maketitle
\thispagestyle{empty}   % quita el nro en la primer pagina
\pagebreak

\begin{abstract}
Este trabajo pr\'actico nro.~0 busca familiarizar a los alumnos con las herramientas de software que se usar\'an a lo largo de la cursada. Se implementar\'a un programa en C (y su correspondiente documentaci\'on) que resuelver\'a el problema piloto que se presentar\'a m\'as abajo. 

Se mostrar\'a el c\'odigo C y se obtendr\'a el c\'odigo Assembly generado por el compilador \textit{gcc}.

El programa correr\'a tanto en NetBSD/pmax, usando el simulador GXemul provisto por la c\'atedra, como en la versi\'on de Linux (Knoppix, RedHat, Debian, Ubuntu) usada para correr el simulador, Linux/i386.

\end{abstract} 
\pagebreak


\section{Introducci\'on}

El programa, a escribir en lenguaje C, deber\'a multiplicar matrices de n\'umeros reales, representados en punto flotante de doble precisi\'on. Las matrices a multiplicar ingresar\'an por entrada est\'andar (\textit{stdin}), donde cada l\'inea describe una matriz completa en formato de texto.
\newline \newline
NxM a1,1 a1,2 ... a1,M a2,1 a2,2 ... a2,M ... aN,1 aN,2 ... aN,M
\newline \newline
La l\'inea anterior representa a la matriz A de N filas y M columnas. Los elementos de la matriz A son los a x,y , siendo x e y los indices de fila y columna respectivamente 1 . El fin de l\'inea es el caracter newline. Los componentes de la l\'nea est\'an separados entre s\'i por uno o m\'as espacios. El formato de los n\'umeros en punto flotante son los que corresponden al especificador de conversi\'on g de printf.

Por ejemplo, dada la siguiente matriz:
\newline
1 2 3
\newline
4 5 6
\newline \newline
Su representaci\'on ser\'ia:
\newline
2x3 1 2 3 4 5 6
\newline \newline
Por cada par de matrices que se presenten en su entrada, el programa deber\'a multiplicarlas y presentar el resultado por su salida es\'andar (stdout) en el mismo formato presentado anteriormente, hasta que llegue al final del archivo de entrada (EOF). Ante un error, el progama
deber\'a informar la situaci\'on inmediatamente (por stderr) y detener su ejecuci\'on. Tener en cuenta que tambi\'en se condidera un error que a la entrada se presenten matrices de dimensiones incompatibles entre s\'i para su multiplicaci\'on.

\pagebreak


\section{Documentaci\'on}

\subsection{Compilaci\'on}
El programa se compilar\'a con la siguiente instrucci\'on:
\begin{center} 
\fbox{gcc -Wall -O0 -o tp0 tp0.c}
\end{center}

Los tests se ejecutar\'an con el siguiente script:
\begin{center} 
\fbox{./tests.sh}
\end{center}

Ejemplo de la salida de ejecuci\'on:
\begin{verbatim}
$ ./tests.sh
	Tests #0 success_normal: 		OK
	Tests #1 sucess_espacios: 		OK
	Tests #2 error_dimension: 		OK
	Tests #3 error_max_dimension: 	OK
	Tests #4 error_max_line: 		OK
	Tests #5 error_matriz1: 			OK
	Tests #6 error_matriz2: 			OK
\end{verbatim}

\subsection{Utilizaci\'on}
Opciones de ejecuci\'on:
\begin{verbatim}
	-h, --help		Print this information.
	-V, --version		Print version and quit.
\end{verbatim}

Ejemplos de ejecuci\'on:
\begin{verbatim}
Examples:
	./tp0 -h
	./tp0 -V
	./tp0 < in_file > out_file
	./tp0 < in.txt > out.txt
	cat in.txt | tp0 > out.txt

\end{verbatim}

A continuaci\'on se presenta un ejemplo de prueba:
\begin{verbatim}
	$ cat in.txt
	2x3 1 2 3 4 5 6.1
	3x2 1 0 0 0 0 1
	3x3 1 2 3 4 5 6.1 3 2 1
	3x1 1 1 0

	$ cat in.txt | ./tp0
	2x2 1 3 4 6.1
	3x1 3 9 5
\end{verbatim}
\pagebreak


\section{Casos de pruebas}
Se crearon los siguientes casos de pruebas:
\begin{enumerate}
\item Caso normal.	
\item Utilizando espacios entre elementos de matriz.
\item Dimensiones de matrices incompatibles.
\item Superaci\'on de caracteres m\'aximos para dimensi\'on.
\item Superaci\'on de caracteres m\'aximos para matriz.
\item Elementos de m\'as con respecto a la dimensi\'on.
\item Elementos de menos con respecto a la dimensi\'on.
\end{enumerate}				
\pagebreak


\section{C\'odigo fuente}

\begin{lstlisting}
#include <stdio.h>
#include <stdlib.h>
#include <getopt.h>
#include <string.h>

#define MAX_LINE_LENGTH 512
#define MAX_DIMENSION_LENGTH 4
#define EXIT_OK 0
#define EXIT_ERROR 1

enum ACCION {
	EMPTY,
	HELP,
	VERSION,
	MULTIPLICAR,
	ERROR
};

void multiplicarMatriz();
enum ACCION procesarArgumentos(int argc, char** argv);

int main(int argc, char **argv) {
	enum ACCION comando = procesarArgumentos(argc, argv);
	switch (comando) {
		case HELP:
			printf("Usage:\n"
			"\ttp0 -h\n"
			"\ttp0 -V\n"
			"\ttp0 < in_file > out_file\n"
			"Options:\n"
			"\t-V, --version\tPrint version and quit.\n"
			"\t-h, --help\tPrint this information.\n"
			"Examples:\n"
			"\ttp0 < in.txt > out.txt\n"
			"\tcat in.txt | tp0 > out.txt\n");
			break;
		case VERSION:
			printf("tp0 v0.1\n");
			break;
		case MULTIPLICAR:
			multiplicarMatriz();
			break;
		case ERROR:
		default:
			return EXIT_ERROR;
	}
	return EXIT_OK;
}

void imprimirMatriz(double** matriz, int filas, int columnas) {
	printf("%dx%d ", filas, columnas);
	int i, j;
	for (i = 0; i < filas; ++i) {
		for (j = 0; j < columnas; ++j) {
			printf("%g ", matriz[i][j]);
		}
	}
	printf("\n");
}

void leerDimension(int* filas, int* columnas) {
	char c;
	char* buffer = (char *) malloc(sizeof(char) * MAX_DIMENSION_LENGTH);
	int i = 0, total = 0;
	while((c = fgetc(stdin)) != EOF) {
		
		if (c == 'x') {
			*filas = atoi(buffer);
			memset(buffer,0,strlen(buffer));
			i = 0;
			continue;
		}

		if(c == ' ') {
			*columnas = atoi(buffer);
			break;
		}

		if(total >= MAX_DIMENSION_LENGTH) {
			fprintf(stderr, "%s\n", "Tamano de buffer superado para la dimension.");
			free(buffer);
			exit(EXIT_ERROR);
		}

		buffer[i] = c;
		i++;
		total++;
	}
	free(buffer);

	if (c == EOF)
		exit(EXIT_OK);
}

void leerMatriz(double** matriz, int filas, int columnas) {
	char c;
	char* buffer = (char *) malloc(sizeof(char) * MAX_LINE_LENGTH);
	int i = 0, total = 0, elementos = 0, fila = 0, columna = 0;
	while((c = fgetc(stdin)) != EOF) {
		
		total++;
		
		if (elementos >= filas * columnas) {
			fprintf(stderr, "%s\n", "Dimension no compatible con datos de matriz.");
			free(buffer);
			exit(EXIT_ERROR);
		}

		if (c == ' ' || c == '\n') {
			// elimino espacios consecutivos
			if (strlen(buffer) == 0)
				continue;

			elementos++;
			matriz[fila][columna] = atof(buffer);
			memset(buffer,0,strlen(buffer));
			i = 0;
			if (columna >= columnas - 1) {
				fila++;
				columna = 0;
			} else {
				columna++;
			}

			if(c == '\n')
				break;

			continue;
		}

		if(total >= MAX_LINE_LENGTH) {
			fprintf(stderr, "%s\n", "Tamano de buffer superado para la matriz.");
			free(buffer);
			exit(EXIT_ERROR);
		}

		buffer[i] = c;
		i++;
	}
	free(buffer);
	
	if (elementos < filas * columnas) {
		fprintf(stderr, "%s\n", "Dimension no compatible con datos de matriz.");
		exit(EXIT_ERROR);
	}
	

	if (c == EOF)
		exit(EXIT_OK);
}

double** crearMatriz(int filas, int columnas) {
	double** matriz = (double**) malloc(sizeof(double*) * filas);
	if (matriz == NULL) {
		fprintf(stderr, "%s\n", "Memoria insuficiente.");
		exit(EXIT_ERROR);
	}
	int i;
	for(i = 0; i < filas; ++i)
		matriz[i] = (double*) malloc(sizeof(double) * columnas);

	if (matriz[i - 1] == NULL) {
		fprintf(stderr, "%s\n", "Memoria insuficiente.");
		exit(EXIT_ERROR);
	}
	return matriz;
}

void multiplicarMatriz() {
	
	while (1) {
		int filasA, columnasA;
		leerDimension(&filasA, &columnasA);
		double** matrizA = crearMatriz(filasA, columnasA);
		leerMatriz(matrizA, filasA, columnasA);

		int filasB, columnasB;
		leerDimension(&filasB, &columnasB);
		double** matrizB = crearMatriz(filasB, columnasB);
		leerMatriz(matrizB, filasB, columnasB);

		if (columnasA != filasB) {
			fprintf(stderr, "%s\n", "Dimensiones incorrectas de matrices.");
			free(matrizA);
			free(matrizB);
			exit(EXIT_ERROR);
		}

		double** matrizC = crearMatriz(filasA, columnasB);
		int i, j, k;
		for (i = 0; i < filasA; ++i){
		  	for (j = 0 ; j < columnasB ; ++j){
		      	matrizC[i][j]=0;
		      	for ( k = 0; k < columnasA; ++k){
		        	matrizC[i][j] = (matrizC[i][j] + (matrizA[i][k] * matrizB[k][j]));
		    	}
			}
		}
		imprimirMatriz(matrizC, filasA, columnasB);

		free(matrizA);
		free(matrizB);
		free(matrizC);
	}
}

enum ACCION procesarArgumentos(int argc, char** argv) {
	// valores por defecto
	enum ACCION comando = EMPTY;
	
	 /* La funcion getopt obtiene el siguiente argumento especificado por argc y argv
	 * mas info: http://www.gnu.org/software/libc/manual/html_node/Using-Getopt.html#Using-Getopt
	 * La cadena "hVbri:o:" indica que h, V no tienen argumentos.
	 */
	int c;
	while ((c = getopt(argc, argv, "hV")) != -1) {
		switch (c) {
			case 'h':
				comando = HELP;
				break;
			case 'V':
				comando = VERSION;
				break;
			default:
				comando = ERROR;
				break;
		}
	}

	if (comando == EMPTY)
		comando = MULTIPLICAR;

	return comando;
}
\end{lstlisting}
\pagebreak


\section{C\'odigo $MIPS^{TM}$}
\begin{lstlisting}
	.file	1 "tp0.c"
	.section .mdebug.abi32
	.previous
	.abicalls
	.rdata
	.align	2
$LC0:
	.ascii	"Usage:\n"
	.ascii	"\ttp0 -h\n"
	.ascii	"\ttp0 -V\n"
	.ascii	"\ttp0 < in_file > out_file\n"
	.ascii	"Options:\n"
	.ascii	"\t-V, --version\tPrint version and quit.\n"
	.ascii	"\t-h, --help\tPrint this information.\n"
	.ascii	"Examples:\n"
	.ascii	"\ttp0 < in.txt > out.txt\n"
	.ascii	"\tcat in.txt | tp0 > out.txt\n\000"
	.align	2
$LC1:
	.ascii	"tp0 v0.1\n\000"
	.text
	.align	2
	.globl	main
	.ent	main
main:
	.frame	$fp,56,$ra		# vars= 16, regs= 3/0, args= 16, extra= 8
	.mask	0xd0000000,-8
	.fmask	0x00000000,0
	.set	noreorder
	.cpload	$t9
	.set	reorder
	subu	$sp,$sp,56
	.cprestore 16
	sw	$ra,48($sp)
	sw	$fp,44($sp)
	sw	$gp,40($sp)
	move	$fp,$sp
	sw	$a0,56($fp)
	sw	$a1,60($fp)
	lw	$a0,56($fp)
	lw	$a1,60($fp)
	la	$t9,procesarArgumentos
	jal	$ra,$t9
	sw	$v0,24($fp)
	lw	$v0,24($fp)
	sw	$v0,32($fp)
	li	$v0,2			# 0x2
	lw	$v1,32($fp)
	beq	$v1,$v0,$L20
	lw	$v1,32($fp)
	sltu	$v0,$v1,3
	beq	$v0,$zero,$L25
	li	$v0,1			# 0x1
	lw	$v1,32($fp)
	beq	$v1,$v0,$L19
	b	$L23
$L25:
	li	$v0,3			# 0x3
	lw	$v1,32($fp)
	beq	$v1,$v0,$L21
	b	$L23
$L19:
	la	$a0,$LC0
	la	$t9,printf
	jal	$ra,$t9
	b	$L18
$L20:
	la	$a0,$LC1
	la	$t9,printf
	jal	$ra,$t9
	b	$L18
$L21:
	la	$t9,multiplicarMatriz
	jal	$ra,$t9
	b	$L18
$L23:
	li	$v0,1			# 0x1
	sw	$v0,28($fp)
	b	$L17
$L18:
	sw	$zero,28($fp)
$L17:
	lw	$v0,28($fp)
	move	$sp,$fp
	lw	$ra,48($sp)
	lw	$fp,44($sp)
	addu	$sp,$sp,56
	j	$ra
	.end	main
	.size	main, .-main
	.rdata
	.align	2
$LC2:
	.ascii	"%dx%d \000"
	.align	2
$LC3:
	.ascii	"%g \000"
	.align	2
$LC4:
	.ascii	"\n\000"
	.text
	.align	2
	.globl	imprimirMatriz
	.ent	imprimirMatriz
imprimirMatriz:
	.frame	$fp,48,$ra		# vars= 8, regs= 3/0, args= 16, extra= 8
	.mask	0xd0000000,-8
	.fmask	0x00000000,0
	.set	noreorder
	.cpload	$t9
	.set	reorder
	subu	$sp,$sp,48
	.cprestore 16
	sw	$ra,40($sp)
	sw	$fp,36($sp)
	sw	$gp,32($sp)
	move	$fp,$sp
	sw	$a0,48($fp)
	sw	$a1,52($fp)
	sw	$a2,56($fp)
	la	$a0,$LC2
	lw	$a1,52($fp)
	lw	$a2,56($fp)
	la	$t9,printf
	jal	$ra,$t9
	sw	$zero,24($fp)
$L27:
	lw	$v0,24($fp)
	lw	$v1,52($fp)
	slt	$v0,$v0,$v1
	bne	$v0,$zero,$L30
	b	$L28
$L30:
	sw	$zero,28($fp)
$L31:
	lw	$v0,28($fp)
	lw	$v1,56($fp)
	slt	$v0,$v0,$v1
	bne	$v0,$zero,$L34
	b	$L29
$L34:
	lw	$v0,24($fp)
	sll	$v1,$v0,2
	lw	$v0,48($fp)
	addu	$a0,$v1,$v0
	lw	$v0,28($fp)
	sll	$v1,$v0,3
	lw	$v0,0($a0)
	addu	$v0,$v1,$v0
	la	$a0,$LC3
	lw	$a2,0($v0)
	lw	$a3,4($v0)
	la	$t9,printf
	jal	$ra,$t9
	lw	$v0,28($fp)
	addu	$v0,$v0,1
	sw	$v0,28($fp)
	b	$L31
$L29:
	lw	$v0,24($fp)
	addu	$v0,$v0,1
	sw	$v0,24($fp)
	b	$L27
$L28:
	la	$a0,$LC4
	la	$t9,printf
	jal	$ra,$t9
	move	$sp,$fp
	lw	$ra,40($sp)
	lw	$fp,36($sp)
	addu	$sp,$sp,48
	j	$ra
	.end	imprimirMatriz
	.size	imprimirMatriz, .-imprimirMatriz
	.rdata
	.align	2
$LC5:
	.ascii	"%s\n\000"
	.align	2
$LC6:
	.ascii	"Tamano de buffer superado para la dimension.\000"
	.text
	.align	2
	.globl	leerDimension
	.ent	leerDimension
leerDimension:
	.frame	$fp,56,$ra		# vars= 16, regs= 3/0, args= 16, extra= 8
	.mask	0xd0000000,-8
	.fmask	0x00000000,0
	.set	noreorder
	.cpload	$t9
	.set	reorder
	subu	$sp,$sp,56
	.cprestore 16
	sw	$ra,48($sp)
	sw	$fp,44($sp)
	sw	$gp,40($sp)
	move	$fp,$sp
	sw	$a0,56($fp)
	sw	$a1,60($fp)
	li	$a0,4			# 0x4
	la	$t9,malloc
	jal	$ra,$t9
	sw	$v0,28($fp)
	sw	$zero,32($fp)
	sw	$zero,36($fp)
$L36:
	la	$a0,__sF
	la	$t9,fgetc
	jal	$ra,$t9
	sb	$v0,24($fp)
	lbu	$v0,24($fp)
	sll	$v0,$v0,24
	sra	$v1,$v0,24
	li	$v0,-1			# 0xffffffffffffffff
	bne	$v1,$v0,$L38
	b	$L37
$L38:
	lb	$v1,24($fp)
	li	$v0,120			# 0x78
	bne	$v1,$v0,$L39
	lw	$a0,28($fp)
	la	$t9,atoi
	jal	$ra,$t9
	lw	$v1,56($fp)
	sw	$v0,0($v1)
	lw	$a0,28($fp)
	la	$t9,strlen
	jal	$ra,$t9
	lw	$a0,28($fp)
	move	$a1,$zero
	move	$a2,$v0
	la	$t9,memset
	jal	$ra,$t9
	sw	$zero,32($fp)
	b	$L36
$L39:
	lb	$v1,24($fp)
	li	$v0,32			# 0x20
	bne	$v1,$v0,$L40
	lw	$a0,28($fp)
	la	$t9,atoi
	jal	$ra,$t9
	move	$v1,$v0
	lw	$v0,60($fp)
	sw	$v1,0($v0)
	b	$L37
$L40:
	lw	$v0,36($fp)
	slt	$v0,$v0,4
	bne	$v0,$zero,$L41
	la	$a0,__sF+176
	la	$a1,$LC5
	la	$a2,$LC6
	la	$t9,fprintf
	jal	$ra,$t9
	lw	$a0,28($fp)
	la	$t9,free
	jal	$ra,$t9
	li	$a0,1			# 0x1
	la	$t9,exit
	jal	$ra,$t9
$L41:
	lw	$v1,28($fp)
	lw	$v0,32($fp)
	addu	$v1,$v1,$v0
	lbu	$v0,24($fp)
	sb	$v0,0($v1)
	lw	$v0,32($fp)
	addu	$v0,$v0,1
	sw	$v0,32($fp)
	lw	$v0,36($fp)
	addu	$v0,$v0,1
	sw	$v0,36($fp)
	b	$L36
$L37:
	lw	$a0,28($fp)
	la	$t9,free
	jal	$ra,$t9
	lb	$v1,24($fp)
	li	$v0,-1			# 0xffffffffffffffff
	bne	$v1,$v0,$L35
	move	$a0,$zero
	la	$t9,exit
	jal	$ra,$t9
$L35:
	move	$sp,$fp
	lw	$ra,48($sp)
	lw	$fp,44($sp)
	addu	$sp,$sp,56
	j	$ra
	.end	leerDimension
	.size	leerDimension, .-leerDimension
	.rdata
	.align	2
$LC7:
	.ascii	"Dimension no compatible con datos de matriz.\000"
	.align	2
$LC8:
	.ascii	"Tamano de buffer superado para la matriz.\000"
	.text
	.align	2
	.globl	leerMatriz
	.ent	leerMatriz
leerMatriz:
	.frame	$fp,72,$ra		# vars= 32, regs= 3/0, args= 16, extra= 8
	.mask	0xd0000000,-8
	.fmask	0x00000000,0
	.set	noreorder
	.cpload	$t9
	.set	reorder
	subu	$sp,$sp,72
	.cprestore 16
	sw	$ra,64($sp)
	sw	$fp,60($sp)
	sw	$gp,56($sp)
	move	$fp,$sp
	sw	$a0,72($fp)
	sw	$a1,76($fp)
	sw	$a2,80($fp)
	li	$a0,512			# 0x200
	la	$t9,malloc
	jal	$ra,$t9
	sw	$v0,28($fp)
	sw	$zero,32($fp)
	sw	$zero,36($fp)
	sw	$zero,40($fp)
	sw	$zero,44($fp)
	sw	$zero,48($fp)
$L44:
	la	$a0,__sF
	la	$t9,fgetc
	jal	$ra,$t9
	sb	$v0,24($fp)
	lbu	$v0,24($fp)
	sll	$v0,$v0,24
	sra	$v1,$v0,24
	li	$v0,-1			# 0xffffffffffffffff
	bne	$v1,$v0,$L46
	b	$L45
$L46:
	lw	$v0,36($fp)
	addu	$v0,$v0,1
	sw	$v0,36($fp)
	lw	$v1,76($fp)
	lw	$v0,80($fp)
	mult	$v1,$v0
	mflo	$v1
	lw	$v0,40($fp)
	slt	$v0,$v0,$v1
	bne	$v0,$zero,$L47
	la	$a0,__sF+176
	la	$a1,$LC5
	la	$a2,$LC7
	la	$t9,fprintf
	jal	$ra,$t9
	lw	$a0,28($fp)
	la	$t9,free
	jal	$ra,$t9
	li	$a0,1			# 0x1
	la	$t9,exit
	jal	$ra,$t9
$L47:
	lb	$v1,24($fp)
	li	$v0,32			# 0x20
	beq	$v1,$v0,$L49
	lb	$v1,24($fp)
	li	$v0,10			# 0xa
	beq	$v1,$v0,$L49
	b	$L48
$L49:
	lw	$v0,28($fp)
	lb	$v0,0($v0)
	bne	$v0,$zero,$L50
	b	$L44
$L50:
	lw	$v0,40($fp)
	addu	$v0,$v0,1
	sw	$v0,40($fp)
	lw	$a0,28($fp)
	la	$t9,atof
	jal	$ra,$t9
	lw	$v0,44($fp)
	sll	$v1,$v0,2
	lw	$v0,72($fp)
	addu	$a0,$v1,$v0
	lw	$v0,48($fp)
	sll	$v1,$v0,3
	lw	$v0,0($a0)
	addu	$v0,$v1,$v0
	s.d	$f0,0($v0)
	lw	$a0,28($fp)
	la	$t9,strlen
	jal	$ra,$t9
	lw	$a0,28($fp)
	move	$a1,$zero
	move	$a2,$v0
	la	$t9,memset
	jal	$ra,$t9
	sw	$zero,32($fp)
	lw	$v0,80($fp)
	addu	$v1,$v0,-1
	lw	$v0,48($fp)
	slt	$v0,$v0,$v1
	bne	$v0,$zero,$L51
	lw	$v0,44($fp)
	addu	$v0,$v0,1
	sw	$v0,44($fp)
	sw	$zero,48($fp)
	b	$L52
$L51:
	lw	$v0,48($fp)
	addu	$v0,$v0,1
	sw	$v0,48($fp)
$L52:
	lb	$v1,24($fp)
	li	$v0,10			# 0xa
	bne	$v1,$v0,$L44
	b	$L45
$L48:
	lw	$v0,36($fp)
	slt	$v0,$v0,512
	bne	$v0,$zero,$L54
	la	$a0,__sF+176
	la	$a1,$LC5
	la	$a2,$LC8
	la	$t9,fprintf
	jal	$ra,$t9
	lw	$a0,28($fp)
	la	$t9,free
	jal	$ra,$t9
	li	$a0,1			# 0x1
	la	$t9,exit
	jal	$ra,$t9
$L54:
	lw	$v1,28($fp)
	lw	$v0,32($fp)
	addu	$v1,$v1,$v0
	lbu	$v0,24($fp)
	sb	$v0,0($v1)
	lw	$v0,32($fp)
	addu	$v0,$v0,1
	sw	$v0,32($fp)
	b	$L44
$L45:
	lw	$a0,28($fp)
	la	$t9,free
	jal	$ra,$t9
	lw	$v1,76($fp)
	lw	$v0,80($fp)
	mult	$v1,$v0
	mflo	$v1
	lw	$v0,40($fp)
	slt	$v0,$v0,$v1
	beq	$v0,$zero,$L55
	la	$a0,__sF+176
	la	$a1,$LC5
	la	$a2,$LC7
	la	$t9,fprintf
	jal	$ra,$t9
	li	$a0,1			# 0x1
	la	$t9,exit
	jal	$ra,$t9
$L55:
	lb	$v1,24($fp)
	li	$v0,-1			# 0xffffffffffffffff
	bne	$v1,$v0,$L43
	move	$a0,$zero
	la	$t9,exit
	jal	$ra,$t9
$L43:
	move	$sp,$fp
	lw	$ra,64($sp)
	lw	$fp,60($sp)
	addu	$sp,$sp,72
	j	$ra
	.end	leerMatriz
	.size	leerMatriz, .-leerMatriz
	.rdata
	.align	2
$LC9:
	.ascii	"Memoria insuficiente.\000"
	.text
	.align	2
	.globl	crearMatriz
	.ent	crearMatriz
crearMatriz:
	.frame	$fp,48,$ra		# vars= 8, regs= 4/0, args= 16, extra= 8
	.mask	0xd0010000,-4
	.fmask	0x00000000,0
	.set	noreorder
	.cpload	$t9
	.set	reorder
	subu	$sp,$sp,48
	.cprestore 16
	sw	$ra,44($sp)
	sw	$fp,40($sp)
	sw	$gp,36($sp)
	sw	$s0,32($sp)
	move	$fp,$sp
	sw	$a0,48($fp)
	sw	$a1,52($fp)
	lw	$v0,48($fp)
	sll	$v0,$v0,2
	move	$a0,$v0
	la	$t9,malloc
	jal	$ra,$t9
	sw	$v0,24($fp)
	lw	$v0,24($fp)
	bne	$v0,$zero,$L58
	la	$a0,__sF+176
	la	$a1,$LC5
	la	$a2,$LC9
	la	$t9,fprintf
	jal	$ra,$t9
	li	$a0,1			# 0x1
	la	$t9,exit
	jal	$ra,$t9
$L58:
	sw	$zero,28($fp)
$L59:
	lw	$v0,28($fp)
	lw	$v1,48($fp)
	slt	$v0,$v0,$v1
	bne	$v0,$zero,$L62
	b	$L60
$L62:
	lw	$v0,28($fp)
	sll	$v1,$v0,2
	lw	$v0,24($fp)
	addu	$s0,$v1,$v0
	lw	$v0,52($fp)
	sll	$v0,$v0,3
	move	$a0,$v0
	la	$t9,malloc
	jal	$ra,$t9
	sw	$v0,0($s0)
	lw	$v0,28($fp)
	addu	$v0,$v0,1
	sw	$v0,28($fp)
	b	$L59
$L60:
	lw	$v0,28($fp)
	sll	$v1,$v0,2
	lw	$v0,24($fp)
	addu	$v0,$v1,$v0
	addu	$v0,$v0,-4
	lw	$v0,0($v0)
	bne	$v0,$zero,$L63
	la	$a0,__sF+176
	la	$a1,$LC5
	la	$a2,$LC9
	la	$t9,fprintf
	jal	$ra,$t9
	li	$a0,1			# 0x1
	la	$t9,exit
	jal	$ra,$t9
$L63:
	lw	$v0,24($fp)
	move	$sp,$fp
	lw	$ra,44($sp)
	lw	$fp,40($sp)
	lw	$s0,32($sp)
	addu	$sp,$sp,48
	j	$ra
	.end	crearMatriz
	.size	crearMatriz, .-crearMatriz
	.rdata
	.align	2
$LC10:
	.ascii	"Dimensiones incorrectas de matrices.\000"
	.text
	.align	2
	.globl	multiplicarMatriz
	.ent	multiplicarMatriz
multiplicarMatriz:
	.frame	$fp,80,$ra		# vars= 40, regs= 3/0, args= 16, extra= 8
	.mask	0xd0000000,-8
	.fmask	0x00000000,0
	.set	noreorder
	.cpload	$t9
	.set	reorder
	subu	$sp,$sp,80
	.cprestore 16
	sw	$ra,72($sp)
	sw	$fp,68($sp)
	sw	$gp,64($sp)
	move	$fp,$sp
	.set	noreorder
	nop
	.set	reorder
$L65:
	addu	$v0,$fp,28
	addu	$a0,$fp,24
	move	$a1,$v0
	la	$t9,leerDimension
	jal	$ra,$t9
	lw	$a0,24($fp)
	lw	$a1,28($fp)
	la	$t9,crearMatriz
	jal	$ra,$t9
	sw	$v0,32($fp)
	lw	$a0,32($fp)
	lw	$a1,24($fp)
	lw	$a2,28($fp)
	la	$t9,leerMatriz
	jal	$ra,$t9
	addu	$v0,$fp,36
	addu	$v1,$fp,40
	move	$a0,$v0
	move	$a1,$v1
	la	$t9,leerDimension
	jal	$ra,$t9
	lw	$a0,36($fp)
	lw	$a1,40($fp)
	la	$t9,crearMatriz
	jal	$ra,$t9
	sw	$v0,44($fp)
	lw	$a0,44($fp)
	lw	$a1,36($fp)
	lw	$a2,40($fp)
	la	$t9,leerMatriz
	jal	$ra,$t9
	lw	$v1,28($fp)
	lw	$v0,36($fp)
	beq	$v1,$v0,$L68
	la	$a0,__sF+176
	la	$a1,$LC5
	la	$a2,$LC10
	la	$t9,fprintf
	jal	$ra,$t9
	lw	$a0,32($fp)
	la	$t9,free
	jal	$ra,$t9
	lw	$a0,44($fp)
	la	$t9,free
	jal	$ra,$t9
	li	$a0,1			# 0x1
	la	$t9,exit
	jal	$ra,$t9
$L68:
	lw	$a0,24($fp)
	lw	$a1,40($fp)
	la	$t9,crearMatriz
	jal	$ra,$t9
	sw	$v0,48($fp)
	sw	$zero,52($fp)
$L69:
	lw	$v0,52($fp)
	lw	$v1,24($fp)
	slt	$v0,$v0,$v1
	bne	$v0,$zero,$L72
	b	$L70
$L72:
	sw	$zero,56($fp)
$L73:
	lw	$v0,56($fp)
	lw	$v1,40($fp)
	slt	$v0,$v0,$v1
	bne	$v0,$zero,$L76
	b	$L71
$L76:
	lw	$v0,52($fp)
	sll	$v1,$v0,2
	lw	$v0,48($fp)
	addu	$a0,$v1,$v0
	lw	$v0,56($fp)
	sll	$v1,$v0,3
	lw	$v0,0($a0)
	addu	$v0,$v1,$v0
	sw	$zero,0($v0)
	sw	$zero,4($v0)
	sw	$zero,60($fp)
$L77:
	lw	$v0,60($fp)
	lw	$v1,28($fp)
	slt	$v0,$v0,$v1
	bne	$v0,$zero,$L80
	b	$L75
$L80:
	lw	$v0,52($fp)
	sll	$v1,$v0,2
	lw	$v0,48($fp)
	addu	$a0,$v1,$v0
	lw	$v0,56($fp)
	sll	$v1,$v0,3
	lw	$v0,0($a0)
	addu	$a2,$v1,$v0
	lw	$v0,52($fp)
	sll	$v1,$v0,2
	lw	$v0,48($fp)
	addu	$a0,$v1,$v0
	lw	$v0,56($fp)
	sll	$v1,$v0,3
	lw	$v0,0($a0)
	addu	$a3,$v1,$v0
	lw	$v0,52($fp)
	sll	$v1,$v0,2
	lw	$v0,32($fp)
	addu	$a0,$v1,$v0
	lw	$v0,60($fp)
	sll	$v1,$v0,3
	lw	$v0,0($a0)
	addu	$a1,$v1,$v0
	lw	$v0,60($fp)
	sll	$v1,$v0,2
	lw	$v0,44($fp)
	addu	$a0,$v1,$v0
	lw	$v0,56($fp)
	sll	$v1,$v0,3
	lw	$v0,0($a0)
	addu	$v0,$v1,$v0
	l.d	$f2,0($a1)
	l.d	$f0,0($v0)
	mul.d	$f2,$f2,$f0
	l.d	$f0,0($a3)
	add.d	$f0,$f0,$f2
	s.d	$f0,0($a2)
	lw	$v0,60($fp)
	addu	$v0,$v0,1
	sw	$v0,60($fp)
	b	$L77
$L75:
	lw	$v0,56($fp)
	addu	$v0,$v0,1
	sw	$v0,56($fp)
	b	$L73
$L71:
	lw	$v0,52($fp)
	addu	$v0,$v0,1
	sw	$v0,52($fp)
	b	$L69
$L70:
	lw	$a0,48($fp)
	lw	$a1,24($fp)
	lw	$a2,40($fp)
	la	$t9,imprimirMatriz
	jal	$ra,$t9
	lw	$a0,32($fp)
	la	$t9,free
	jal	$ra,$t9
	lw	$a0,44($fp)
	la	$t9,free
	jal	$ra,$t9
	lw	$a0,48($fp)
	la	$t9,free
	jal	$ra,$t9
	b	$L65
	.end	multiplicarMatriz
	.size	multiplicarMatriz, .-multiplicarMatriz
	.rdata
	.align	2
$LC11:
	.ascii	"hV\000"
	.text
	.align	2
	.globl	procesarArgumentos
	.ent	procesarArgumentos
procesarArgumentos:
	.frame	$fp,56,$ra		# vars= 16, regs= 3/0, args= 16, extra= 8
	.mask	0xd0000000,-8
	.fmask	0x00000000,0
	.set	noreorder
	.cpload	$t9
	.set	reorder
	subu	$sp,$sp,56
	.cprestore 16
	sw	$ra,48($sp)
	sw	$fp,44($sp)
	sw	$gp,40($sp)
	move	$fp,$sp
	sw	$a0,56($fp)
	sw	$a1,60($fp)
	sw	$zero,24($fp)
$L82:
	lw	$a0,56($fp)
	lw	$a1,60($fp)
	la	$a2,$LC11
	la	$t9,getopt
	jal	$ra,$t9
	sw	$v0,28($fp)
	lw	$v1,28($fp)
	li	$v0,-1			# 0xffffffffffffffff
	bne	$v1,$v0,$L84
	b	$L83
$L84:
	lw	$v0,28($fp)
	sw	$v0,32($fp)
	li	$v0,86			# 0x56
	lw	$v1,32($fp)
	beq	$v1,$v0,$L87
	li	$v0,104			# 0x68
	lw	$v1,32($fp)
	beq	$v1,$v0,$L86
	b	$L88
$L86:
	li	$v0,1			# 0x1
	sw	$v0,24($fp)
	b	$L82
$L87:
	li	$v0,2			# 0x2
	sw	$v0,24($fp)
	b	$L82
$L88:
	li	$v0,4			# 0x4
	sw	$v0,24($fp)
	b	$L82
$L83:
	lw	$v0,24($fp)
	bne	$v0,$zero,$L90
	li	$v0,3			# 0x3
	sw	$v0,24($fp)
$L90:
	lw	$v0,24($fp)
	move	$sp,$fp
	lw	$ra,48($sp)
	lw	$fp,44($sp)
	addu	$sp,$sp,56
	j	$ra
	.end	procesarArgumentos
	.size	procesarArgumentos, .-procesarArgumentos
	.ident	"GCC: (GNU) 3.3.3 (NetBSD nb3 20040520)"
\end{lstlisting}
\pagebreak

\end{document}
